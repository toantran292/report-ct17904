\phantomsection
\subsection{(10\%) Cài đặt và cấu hình dịch vụ SSH để cho phép điều khiển từ xa server}


\begin{itemize}
  \item [--] Chỉ có thành viên ban giám đốc và các trưởng phòng mới có quyền điều khiển từ xa server. Tài khoản root không được nối kết tới server từ xa.
  \item [--] Chỉ cho phép chứng thực bằng private key, không cho phép chứng thực bằng password. Tạo private/public key cho người dùng Gia Cát Lượng để có SSH tới server.
\end{itemize}

\phantomsection
\subsubsection{Cài đặt dịch vụ SSH}
\figminiskip{./imgs/Hinh-22.png}{fig:installing-ssh}{Cài đặt và kích hoạt dịch vụ openssh-server}


\vspace{0.5cm}
\begin{coding}{Cài đặt và kích hoạt dịch vụ openssh-server}
  sudo dnf install openssh-server -y
  sudo systemctl start sshd
  sudo systemctl enable sshd
\end{coding}

\phantomsection
\subsubsection{Cấu hình chỉ cho phép thành viên trong ban giám đốc và các trưởng phòng mới có quyền điều khiển từ xa}

Ta sẽ cấu hình file \texttt{/etc/ssh/sshd\_config} \textit{(\myref{fig:config-sshd})} để cấu hình chỉ cho phép một nhóm người dùng hoặc người dùng có thể sử dụng dịch vụ SSH.

\figminiskip{./imgs/Hinh-23.png}{fig:config-sshd}{Cấu hình chỉ cho phép \texttt{bangiamdoc} và \texttt{truongphong} sử dụng dịch vụ SSH để điều khiển từ xa}

\begin{itemize}[label={}]
  \item
        \begin{itemize}[label={--}]
          \item \texttt{AllowGroups bangiamdoc truongnhom}: Cho phép nhóm \texttt{bangiamdoc} và nhóm \texttt{truongnhom} sử dụng dịch vụ ssh.
        \end{itemize}
\end{itemize}

Ta cần khởi động lại dịch vụ ssh để áp dụng những thay đổi này (dùng lệnh \texttt{systemctl restart sshd}).

\phantomsection
\subsubsection{Chỉ cho phép chứng thực bằng private key}

Để cấu hình chỉ cho phép chứng thực bằng private key, ta sẽ cấu hình trong file \texttt{/etc/ssh/sshd\_config} \textit{(\myref{fig:config-sshd-2})}

\figminiskip{./imgs/Hinh-24.png}{fig:config-sshd-2}{Cấu hình chỉ cho phép chứng thực bằng private key}

\begin{itemize}[label={}]
  \item
        \begin{itemize}[label={--}]
          \item \texttt{PubkeyAuthentication yes}: Cho phép chứng thực bằng private key.
          \item \texttt{PasswordAuthentication no}: Không cho phép chứng thực bằng password.
        \end{itemize}
\end{itemize}

Ta cần khởi động lại dịch vụ ssh để áp dụng những thay đổi này (dùng lệnh \texttt{systemctl restart sshd}).

Ta sử dụng lệnh \texttt{ssh-keygen} để tạo private/public key.

\figminiskip{./imgs/Hinh-25.png}{fig:using-ssh-keygen}{Tạo public/private key bằng \texttt{ssh-keygen}}

Tiếp theo, ta cần đổi tên của public key thành authenrized\_keys và phân lại quyền
cho file này \textit{(\myref{fig:change-authenrized-keys})}.

\figminiskip{./imgs/Hinh-26.png}{fig:change-authenrized-keys}{Đổi tên và phân quyền cho file public key}



\begin{itemize}[label={--}]
  \item Đổi tên tập tin public key thành \texttt{authorized\_key} cho người dùng \texttt{luong.giacat}.\\
        \begin{coding}[gobble=10]{Đổi tên tập tin public key thành \texttt{authorized\_key} cho người dùng \texttt{luong.giacat}}
          sudo -u luong.giacat mv /home/luong.giacat/.ssh/id_rsa.pub\  /home/luong.giacat/.ssh/authorized_keys
        \end{coding}

  \item Cho phép \texttt{luong.giacat} đọc và ghi vào tập tin \texttt{authorized\_key}.\\
        \begin{coding}[gobble=10]{Cho phép \texttt{luong.giacat} đọc và ghi vào tập tin \texttt{authorized\_key}}
          sudo -u luong.giacat chmod 600 \                          /home/luong.giacat/.ssh/authorized_keys
        \end{coding}
\end{itemize}



