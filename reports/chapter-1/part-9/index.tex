\phantomsection
\subsection{(5\%) Cài đặt và cấu hình tường lửa trên server}

\begin{itemize}[label={--}]
  \item Có thể truy cập các dịch vụ DNS, DHCP, SSH, Web, FTP trên server. Các
        dịch vụ khác KHÔNG cập truy cập được.
  \item Chỉ máy desktop có thể SSH tới server, các máy khác KHÔNG SSH được.
\end{itemize}


\phantomsection
\subsubsection{Cấu hình tường lửa cho phép các dịch vụ DNS, DHCP, SSH, Web, FTP}

Ta cấu hình như sau \textit{(\myref{fig:config-firewall})}

\figminiskip{./imgs/Hinh-41.png}{fig:config-firewall}{Cấu hình tường lửa cho phép các dịch vụ DNS, DHCP, SSH, Web, FTP}

Cụ thể các bước như sau
\begin{enumerate}
  \item Khởi động dịch vụ firewalld\\
        \begin{coding}[gobble=10]{Tạo zone mới có tên là \texttt{services}}
          sudo systemctl start firewalld
        \end{coding}
  \item Tạo một zone mới có tên là \texttt{services}\\
        \begin{coding}[gobble=10]{Tạo zone mới có tên là \texttt{services}}
          sudo firewall-cmd --permanent --new-zone=services
        \end{coding}
  \item Thêm các dịch vụ DNS, DHCP, SSH, Web, FTP vào zone \texttt{services}\\
        \begin{coding}[gobble=10]{Thêm các dịch vụ DNS, DHCP, SSH, Web, FTP vào zone \texttt{services}}
          sudo firewall-cmd --permanent --zone=services \ --add-service=dns
          sudo firewall-cmd --permanent --zone=services \ --add-service=dhcp
          sudo firewall-cmd --permanent --zone=services \ --add-service=ssh
          sudo firewall-cmd --permanent --zone=services \ --add-service=http
          sudo firewall-cmd --permanent --zone=services \ --add-service=ftp
        \end{coding}
  \item Khởi động lại dịch vụ tường lửa để áp dụng những thay đổi này\\
        \begin{coding}[gobble=10]{Khởi động lại firewalld sau khi đổi zone mới}
          sudo systemctl restart firewalld
        \end{coding}
\end{enumerate}

\phantomsection
\subsubsection{Cấu hình chỉ cho phép máy desktop mới có thể SSH tới server}

Ta cấu hình như sau \textit{(\myref{fig:config-firewall-2})}

\figminiskip{./imgs/Hinh-44.png}{fig:config-firewall-2}{Cấu hình chỉ cho phép máy desktop mới có thể SSH tới server}

Cụ thể các bước như sau
\begin{enumerate}
  \item Thêm một rule mới cho phép desktop (10.0.2.50) vào zone có port 22 (port của SSH)\\
        \begin{coding}[gobble=10]{Thêm một rule mới cho phép desktop (10.0.2.50) vào zone có port 22 (port của SSH)}
          sudo firewall-cmd --zone=services --add-rich-rule='rule family="ipv4" source address="10.0.2.50/24" port protocol="tcp" port="22" accept' --permanent
        \end{coding}
  \item Xóa dịch vụ SSH ra khỏi zone\\
        \begin{coding}[gobble=10]{Xóa dịch vụ SSH ra khỏi zone}
          sudo firewall-cmd --permanent --zone=services \ --remove-serivce=ssh
        \end{coding}
  \item Khởi động lại dịch vụ tường lửa để áp dụng những thay đổi này\\
        \begin{coding}[gobble=10]{Khởi động lại firewalld sau khi thêm rule mới}
          sudo systemctl restart firewalld
        \end{coding}
\end{enumerate}

